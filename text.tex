\section{Περιγραφή}

Το παρόν αποτελεί παράδειγμα χρήσης της \LaTeX (και πιο συγκεκριμένα της
XeLaTeX) για τη δημιουργία κειμένων στα οποία χρησιμοποιούνται
ταυτόχρονα ελληνικοί και λατινικοί χαρακτήρες, χωρίς να απαιτείται η
χρήση εντολών για την εναλλαγή της γλώσσας μεταξύ των δύο γλωσσών. Ο
συλλαβισμός (hyphenation) υποστηρίζεται ταυτόχρονα και για τις δύο
γλώσσες.

\begin{figure}
\centering
\includegraphics[width=0.3\textwidth]{latex.jpg}
\caption{Μπορεί να γράφεις βλακείες, αλλά το typesetting είναι απίστευτο!}
\label{fig:latex}
\end{figure}

\subsection{Προϋποθέσεις}

Τα \verb|*.tex| αρχεία θα πρέπει να χρησιμοποιούν κωδικοποίηση UTF-8 και
ο text editor που χρησιμοποιείται να υποστηρίζει και να είναι
ρυθμισμένος ώστε να χρησιμοποιεί αυτή τη κωδικοποίηση χαρακτήρων.

Το παράδειγμα αυτό έχει δοκιμαστεί μόνο σε σύστημα Linux, αλλά πέρα από
τη χρήση του Makefile για τη δημιουργία του τελικού κειμένου σε μορφή
PDF, δεν υπάρχει κάποιος λόγος ώστε να μην μπορεί να χρησιμοποιηθεί και
σε άλλο λειτουργικό σύστημα. Το default font είναι το Linux Libertine.
Σε συστήματα Linux είναι συνήθως προεγκατεστημένο, αλλιώς φροντίστε να
το εγκαταστήσετε. Ή αλλάξτε το στο αρχείο \emph{main.tex} τέλος πάντων.

\subsection{Δημιουργία PDF κλπ}

Τρέχοντας απλά την εντολή \verb|make| δημιουργείται το PDF, με πλήρη
υποστήριξη βιβλιογραφίας. Παράλληλα παρέχονται και οι επιλογές
\verb|make docx| και \verb|make odt| οι οποίες μπορούν να δημιουργήσουν
ένα αρχείο MS Word ή OpenDocument αντίστοιχα. Μην περιμένετε τα
τελευταία να δείχνουν τέλεια όπως το PDF, αλλά αν είστε αναγκασμένοι να
δημιουργήσετε κάτι τέτοιο, είναι μια λύση. Το τελευταίο απαιτεί
εγκατεστημένο το \emph{pandoc}. Σε οποιαδήποτε περίπτωση δημιουργείται
ένα αρχείο με όνομα \emph{output} και την αντίστοιχη επέκταση.

\section{Διαχείριση βιβλιογραφίας}

Το κομμάτι της βιβλιογραφίας έχει χωριστεί σε δύο μέρη, το πρώτο με
τίτλο «Βιβλιογραφικές αναφορές» και το δεύτερο με τίτλο «Διαδικτυακές
αναφορές». Στο δεύτερο μπαίνουν αυτόματα όσες αναφορές είναι του τύπου
\verb|@MISC|, ενώ στο πρώτο όλες οι υπόλοιπες. Στις «Διαδικτυακές
αναφορές» υποστηρίζεται και η προσθήκη της ημερομηνίας προσπέλασης.
Δείτε τα παραδείγματα στο αρχείο της βιβλιογραφίας \emph{main.bib}, στο
οποίο περιλαμβάνονται δυο βιβλία \cite{goossens93}\cite{Syropoulos} και
μια ιστοσελίδα \cite{JABREF} (Συμβουλή: χρησιμοποιήστε ένα εξειδικευμένο
editor για τη βιβλιογραφία, όπως το JabRef \cite{JABREF}). Ταυτόχρονη
χρήση ελληνικών και λατινικών χαρακτήρων υποστηρίζεται φυσικά και στη
βιβλιογραφία.

\begin{table}
\caption{Κι ένας πίνακας για να έχουμε}
\label{tab:sampletable}
\centering
\begin{tabular}{l*{6}{c}r}
Ομάδα                & P & W & D & L & F  & A & Pts \\
\hline
Manchester United    & 6 & 4 & 0 & 2 & 10 & 5 & 12  \\
Celtic               & 6 & 3 & 0 & 3 &  8 & 9 &  9  \\
Benfica              & 6 & 2 & 1 & 3 &  7 & 8 &  7  \\
Κεραυνός Κάτω Μηλιάς & 6 & 2 & 1 & 3 &  5 & 8 &  7  \\
\end{tabular}
\end{table}
